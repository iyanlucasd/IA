\documentclass[12pt]{article}                                                                                                                       
\usepackage{sbc-template}                                                 
\usepackage{graphicx,url}                                                 
\usepackage[utf8]{inputenc}                                               
\usepackage[brazil]{babel}                                                      
\usepackage{graphicx}


\title{Lista 4\\ Inteligência Artificial}
\author{Iyan Lucas Duarte Marques\inst{1}, Samir do Morim Cambraia\inst{1}, Wesley Filemon\inst{1}}

\address{Instituto de Ciências Exatas e Informática - Pontifícia Universidade Católica Minas Gerais (PUC-MG)}

\begin{document}

\maketitle

\section{Questão 01\\
  A partir da base de dados PãoeManteiga Sim.csvdisponível no CANVAS, pede-se:
 }
\begin{table}[h]
	\begin{center}

		\begin{tabular}{| l | l |}
			\hline
			\textbf{Produto} & \textbf{Suporte} \\ \hline
			Leite            & $2/10=0.2$       \\ \hline
			Café             & $3/10=0.3$       \\ \hline
			Cerveja          & $2/10=0.2$       \\ \hline
			Pão              & $5/10=0.5$       \\ \hline
			Manteiga         & $5/10=0.5$       \\ \hline
			Arroz            & $2/10=0.2$       \\ \hline
			Feijão           & $2/10=0.2$       \\ \hline
		\end{tabular}
		\caption{\label{tab:itemset1.1}Itemset = 1}
	\end{center}
\end{table}

\begin{table}[h]
	\begin{center}

		\begin{tabular}{| l | l |}
			\hline
			\textbf{Produto} & \textbf{Suporte} \\ \hline
			Café e Pão       & $3/10=0.3$       \\ \hline
			Café e Manteiga  & $3/10=0.3$       \\ \hline
			Pão e Manteiga   & $4/10=0.4$       \\ \hline
		\end{tabular}
		\caption{\label{tab:itemset1.2}Itemset = 2}
	\end{center}
\end{table}

\begin{table}[h]
	\begin{center}

		\begin{tabular}{| l | l |}
			\hline
			\textbf{Produto}     & \textbf{Suporte} \\ \hline
			Café, Pão e Manteiga & $3/10=0.3$       \\ \hline
		\end{tabular}
		\caption{\label{tab:itemset1.3}Itemset = 3}
	\end{center}
\end{table}
\subsection{Encontre os itensSetse as regras de associação do problema, utilizando-se o algoritmo APRIORI}

\begin{itemize}
	\item \textbf{\textit{[Café, Pão]}}
	      \begin{itemize}
		      \item Se Café → Pão	Confiança= 3/3 = 1
		      \item Se Pão  → Café	Confiança= 3/5 = 0.6
	      \end{itemize}
	\item \textbf{\textit{[Café, Manteiga]}}
	      \begin{itemize}
		      \item Se Café → Manteiga     Confiança= 3/3 = 1
		      \item Se Manteiga → Café	 Confiança= 3/5 = 0.6
	      \end{itemize}
	\item  \textbf{\textit{[Pão, Manteiga]}}
	      \begin{itemize}
		      \item Se Pão → Manteiga	  Confiança= 4/5 = 0.8
		      \item Se Manteiga → Pão	  Confiança= 4/5 = 0.8
	      \end{itemize}
	\item \textbf{\textit{[Café, Pão e Manteiga]}}
	      \begin{itemize}
		      \item Se Café e Pão → Manteiga	Confiança = 3/3 = 1
		      \item Se café e manteiga → Pão	Confiança = 3/3 = 1
		      \item Se Pão e Manteiga → Pão	Confiança = 3/4 = 0,75
		      \item Se Café → Pão e Manteiga	Confiança = 3/3 = 1
		      \item Se Pão → Café e Manteiga	Confiança = 3/5 = 0.6
		      \item Se Manteiga → Café e Pão 	Confiança = 3/5 = 0.6
	      \end{itemize}


\end{itemize}
\subsection{Rode a base de dados no WEKA (ou qualquer ferramenta que você esteja trabalhando), habilite o parâmetro para exibir os ItensSets e verifique se o que você encontrou no item 1 está correto. Utilize os mesmos valores de suporte = 0.3 e confiança de 0.8}
Ao todo foram encontrados 12 tipos de regras em relação a quem levou café e manteiga temos a regra se café → manteiga vai mostrar que todas as pessoas que levaram café também levou manteiga mas apenas 0.6 dos que levaram manteiga levaram café.
Já em pão e manteiga ambos tiveram o mesmo resultado aqueles que levaram pão e manteiga e manteiga e pão tiveram o resultado de 0.8.
Em relação aos três, nos temos que quem levou café e pão também levou manteiga, quem levou café e manteiga também levou pão, e se levou café também levou pão e manteiga, agora quem levou pão e manteiga também levou pão tiveram o resultado de 0,75.
E quem levou pão também levou café e manteiga teve o resultado de 0.6, e quem levou manteiga e também levou café e pão também teve o resultado de 0,6.

\section{Questão 02\\
  A partir da base de dados PãoeManteigaSimNao.csv disponível no CANVAS, pede-se:
 }
\subsection{Realize os mesmos procedimentos da questão anterior e discuta sobre a quantidade de regras encontradas e sobre o tipo de regras}
\begin{table}[h]
	\begin{center}

		\begin{tabular}{| l | l |}
			\hline
			\textbf{Produto} & \textbf{Suporte} \\ \hline
			Leite            & $2/10=0.2$       \\ \hline
			Café             & $3/10=0.3$       \\ \hline
			Cerveja          & $2/10=0.2$       \\ \hline
			Pão              & $5/10=0.5$       \\ \hline
			Manteiga         & $5/10=0.5$       \\ \hline
			Arroz            & $2/10=0.2$       \\ \hline
			Feijão           & $2/10=0.2$       \\ \hline
		\end{tabular}
		\caption{\label{tab:itemset1.1}Itemset = 1}
	\end{center}
\end{table}

\begin{table}[h]
	\begin{center}

		\begin{tabular}{| l | l |}
			\hline
			\textbf{Produto} & \textbf{Suporte} \\ \hline
			Café e Pão       & $3/10=0.3$       \\ \hline
			Café e Manteiga  & $3/10=0.3$       \\ \hline
			Pão e Manteiga   & $4/10=0.4$       \\ \hline
		\end{tabular}
		\caption{\label{tab:itemset1.2}Itemset = 2}
	\end{center}
\end{table}

\begin{table}[h]
	\begin{center}

		\begin{tabular}{| l | l |}
			\hline
			\textbf{Produto}     & \textbf{Suporte} \\ \hline
			Café, Pão e Manteiga & $3/10=0.3$       \\ \hline
		\end{tabular}
		\caption{\label{tab:itemset1.3}Itemset = 3}
	\end{center}
\end{table}

\begin{itemize}
	\item \textbf{\textit{[Café, Pão]}}
	      \begin{itemize}
		      \item Se Café → Pão	Confiança= 3/3 = 1
		      \item Se Pão  → Café	Confiança= 3/5 = 0.6
	      \end{itemize}
	\item \textbf{\textit{[Café, Manteiga]}}
	      \begin{itemize}
		      \item Se Café → Manteiga     Confiança= 3/3 = 1
		      \item Se Manteiga → Café	 Confiança= 3/5 = 0.6
	      \end{itemize}
	\item  \textbf{\textit{[Pão, Manteiga]}}
	      \begin{itemize}
		      \item Se Pão → Manteiga	  Confiança= 4/5 = 0.8
		      \item Se Manteiga → Pão	  Confiança= 4/5 = 0.8
	      \end{itemize}
	\item \textbf{\textit{[Café, Pão e Manteiga]}}
	      \begin{itemize}
		      \item Se Café e Pão → Manteiga	Confiança = 3/3 = 1
		      \item Se café e manteiga → Pão	Confiança = 3/3 = 1
		      \item Se Pão e Manteiga → Pão	Confiança = 3/4 = 0,75
		      \item Se Café → Pão e Manteiga	Confiança = 3/3 = 1
		      \item Se Pão → Café e Manteiga	Confiança = 3/5 = 0.6
		      \item Se Manteiga → Café e Pão 	Confiança = 3/5 = 0.6
	      \end{itemize}


\end{itemize}

Ao todo foram encontrados 12 tipos de regras em relação a quem levou café e manteiga temos a regra se café → manteiga vai mostrar que todas as pessoas que levaram café também levou manteiga mas apenas 0.6 dos que levaram manteiga levaram café.
Já em pão e manteiga ambos tiveram o mesmo resultado aqueles que levaram pão e manteiga e manteiga e pão tiveram o resultado de 0.8.
Em relação aos três, nos temos que quem levou café e pão também levou manteiga, quem levou café e manteiga também levou pão, e se levou café também levou pão e manteiga, agora quem levou pão e manteiga também levou pão tiveram o resultado de 0,75.
E quem levou pão também levou café e manteiga teve o resultado de 0.6, e quem levou manteiga e também levou café e pão também teve o resultado de 0,6.


\section{Questão 03\\
  Considere a seguinte tabela:
 }

\subsection{Crie um arquivo arff ou CSV com esta tabela}
Vide anexo: questao3.csv.

\subsection{Com o método APRIORI, com os demais parâmetros default, na ferramenta Weka, descubra 15 regras}
Apriori
=======

Minimum support: 0.55 (6 instances)
Minimum metric <confidence>: 0.9
Number of cycles performed: 9

Generated sets of large itemsets:

Size of set of large itemsets L(1): 8

Size of set of large itemsets L(2): 13

Size of set of large itemsets L(3): 5

Best rules found:

1. Banana=Sim 7 ==>  Shampoo=Sim 7    <conf:(1)> lift:(1.22) lev:(0.12) [1] conv:(1.27)\\
2. Banana=Sim 7 ==>  Protetor solar=Não 7    <conf:(1)> lift:(1.22) lev:(0.12) [1] conv:(1.27)\\
3.  Pasta de dente=Não 7 ==>  Protetor solar=Não 7    <conf:(1)> lift:(1.22) lev:(0.12) [1] conv:(1.27)\\
4.  Creme para mãos=Não 7 ==>  Protetor solar=Não 7    <conf:(1)> lift:(1.22) lev:(0.12) [1] conv:(1.27)\\
5. Banana=Sim  Protetor solar=Não 7 ==>  Shampoo=Sim 7    <conf:(1)> lift:(1.22) lev:(0.12) [1] conv:(1.27)\\
6. Banana=Sim  Shampoo=Sim 7 ==>  Protetor solar=Não 7    <conf:(1)> lift:(1.22) lev:(0.12) [1] conv:(1.27)\\
7. Banana=Sim 7 ==>  Shampoo=Sim  Protetor solar=Não 7    <conf:(1)> lift:(1.38) lev:(0.17) [1] conv:(1.91)\\
8.  Ervilha=Sim 6 ==>  Pasta de dente=Não 6    <conf:(1)> lift:(1.57) lev:(0.2) [2] conv:(2.18)\\
9.  Ervilha=Sim 6 ==>  Protetor solar=Não 6    <conf:(1)> lift:(1.22) lev:(0.1) [1] conv:(1.09)\\
10.  Pasta de dente=Não  Shampoo=Sim 6 ==>  Protetor solar=Não 6    <conf:(1)> lift:(1.22) lev:(0.1) [1] conv:(1.09)\\

\subsection{Altere a confiança para 0.6 e veja o que ocorre. Registre os resultados dos experimentos.}
Apriori
=======

Minimum support: 0.65 (7 instances)
Minimum metric <confidence>: 0.6
Number of cycles performed: 7

Generated sets of large itemsets:

Size of set of large itemsets L(1): 7

Size of set of large itemsets L(2): 5

Size of set of large itemsets L(3): 1

Best rules found:

 1. Banana=Sim 7 ==>  Shampoo=Sim 7    <conf:(1)> lift:(1.22) lev:(0.12) [1] conv:(1.27)\\
 2. Banana=Sim 7 ==>  Protetor solar=Não 7    <conf:(1)> lift:(1.22) lev:(0.12) [1] conv:(1.27)\\
 3.  Pasta de dente=Não 7 ==>  Protetor solar=Não 7    <conf:(1)> lift:(1.22) lev:(0.12) [1] conv:(1.27)\\
 4.  Creme para mãos=Não 7 ==>  Protetor solar=Não 7    <conf:(1)> lift:(1.22) lev:(0.12) [1] conv:(1.27)\\
 5. Banana=Sim  Protetor solar=Não 7 ==>  Shampoo=Sim 7    <conf:(1)> lift:(1.22) lev:(0.12) [1] conv:(1.27)\\
 6. Banana=Sim  Shampoo=Sim 7 ==>  Protetor solar=Não 7    <conf:(1)> lift:(1.22) lev:(0.12) [1] conv:(1.27)\\
 7. Banana=Sim 7 ==>  Shampoo=Sim  Protetor solar=Não 7    <conf:(1)> lift:(1.38) lev:(0.17) [1] conv:(1.91)\\
 8.  Protetor solar=Não 9 ==>  Shampoo=Sim 8    <conf:(0.89)> lift:(1.09) lev:(0.06) [0] conv:(0.82)\\
 9.  Shampoo=Sim 9 ==>  Protetor solar=Não 8    <conf:(0.89)> lift:(1.09) lev:(0.06) [0] conv:(0.82)\\
10.  Shampoo=Sim  Protetor solar=Não 8 ==> Banana=Sim 7    <conf:(0.88)> lift:(1.38) lev:(0.17) [1] conv:(1.45)\\


\section{Questão 04\\
  Abra o arquivo iris.arff e execute o algoritmo Apriori.
 }
\subsection{Porque ele não está habilitado? }
Porque o algoritmo só funciona com valores nominais.

\subsection{Discretize então os dados numéricos em 5 grupos (bins) (PreProcess/filters/unsupervised/attribute/discretize). Quantas regras foram encontradas? Analise algumas regras}
=== Run information ===

Scheme:       weka.associations.Apriori -N 10 -T 0 -C 0.6 -D 0.05 -U 1.0 -M 0.1 -S -1.0 -c -1
Relation:     iris-weka.filters.unsupervised.attribute.Discretize-B10-M-1.0-Rfirst-last-precision6
Instances:    150
Attributes:   5
              sepallength
              sepalwidth
              petallength
              petalwidth
              class
=== Associator model (full training set) ===


Apriori
=======

Minimum support: 0.2 (30 instances)
Minimum metric <confidence>: 0.6
Number of cycles performed: 16

Generated sets of large itemsets:

Size of set of large itemsets L(1): 6

Size of set of large itemsets L(2): 3

Size of set of large itemsets L(3): 1

Best rules found:

 1. petalwidth='(-inf-0.34]' 41 ==> class=Iris-setosa 41    <conf:(1)> lift:(3) lev:(0.18) [27] conv:(27.33)\\
 2. petallength='(-inf-1.59]' 37 ==> class=Iris-setosa 37    <conf:(1)> lift:(3) lev:(0.16) [24] conv:(24.67)\\
 3. petallength='(-inf-1.59]' petalwidth='(-inf-0.34]' 33 ==> class=Iris-setosa 33    <conf:(1)> lift:(3) lev:(0.15) [22] conv:(22)\\
 4. petallength='(-inf-1.59]' 37 ==> petalwidth='(-inf-0.34]' 33    <conf:(0.89)> lift:(3.26) lev:(0.15) [22] conv:(5.38)\\
 5. petallength='(-inf-1.59]' class=Iris-setosa 37 ==> petalwidth='(-inf-0.34]' 33    <conf:(0.89)> lift:(3.26) lev:(0.15) [22] conv:(5.38\\)
 6. petallength='(-inf-1.59]' 37 ==> petalwidth='(-inf-0.34]' class=Iris-setosa 33    <conf:(0.89)> lift:(3.26) lev:(0.15) [22] conv:(5.38\\)
 7. class=Iris-setosa 50 ==> petalwidth='(-inf-0.34]' 41    <conf:(0.82)> lift:(3) lev:(0.18) [27] conv:(3.63)\\
 8. petalwidth='(-inf-0.34]' 41 ==> petallength='(-inf-1.59]' 33    <conf:(0.8)> lift:(3.26) lev:(0.15) [22] conv:(3.43)\\
 9. petalwidth='(-inf-0.34]' class=Iris-setosa 41 ==> petallength='(-inf-1.59]' 33    <conf:(0.8)> lift:(3.26) lev:(0.15) [22] conv:(3.43)\\
10. petalwidth='(-inf-0.34]' 41 ==> petallength='(-inf-1.59]' class=Iris-setosa 33    <conf:(0.8)> lift:(3.26) lev:(0.15) [22] conv:(3.43)\\


\subsection{Refaça a discretização, agora considerando 3 grupos . Qual o resultado?}
=== Run information ===

Scheme:       weka.associations.Apriori -N 10 -T 0 -C 0.6 -D 0.05 -U 1.0 -M 0.1 -S -1.0 -c -1
Relation:     iris-weka.filters.unsupervised.attribute.Discretize-B10-M-1.0-Rfirst-last-precision6
Instances:    150
Attributes:   5
              sepallength
              sepalwidth
              petallength
              petalwidth
              class
=== Associator model (full training set) ===


Apriori
=======

Minimum support: 0.2 (30 instances)
Minimum metric <confidence>: 0.6
Number of cycles performed: 16

Generated sets of large itemsets:

Size of set of large itemsets L(1): 6

Size of set of large itemsets L(2): 3

Size of set of large itemsets L(3): 1

Best rules found:

 1. petalwidth='(-inf-0.34]' 41 ==> class=Iris-setosa 41    <conf:(1)> lift:(3) lev:(0.18) [27] conv:(27.33)\\
 2. petallength='(-inf-1.59]' 37 ==> class=Iris-setosa 37    <conf:(1)> lift:(3) lev:(0.16) [24] conv:(24.67)\\
 3. petallength='(-inf-1.59]' petalwidth='(-inf-0.34]' 33 ==> class=Iris-setosa 33    <conf:(1)> lift:(3) lev:(0.15) [22] conv:(22)\\
 4. petallength='(-inf-1.59]' 37 ==> petalwidth='(-inf-0.34]' 33    <conf:(0.89)> lift:(3.26) lev:(0.15) [22] conv:(5.38)\\
 5. petallength='(-inf-1.59]' class=Iris-setosa 37 ==> petalwidth='(-inf-0.34]' 33    <conf:(0.89)> lift:(3.26) lev:(0.15) [22] conv:(5.38\\)
 6. petallength='(-inf-1.59]' 37 ==> petalwidth='(-inf-0.34]' class=Iris-setosa 33    <conf:(0.89)> lift:(3.26) lev:(0.15) [22] conv:(5.38\\)
 7. class=Iris-setosa 50 ==> petalwidth='(-inf-0.34]' 41    <conf:(0.82)> lift:(3) lev:(0.18) [27] conv:(3.63)\\
 8. petalwidth='(-inf-0.34]' 41 ==> petallength='(-inf-1.59]' 33    <conf:(0.8)> lift:(3.26) lev:(0.15) [22] conv:(3.43)\\
 9. petalwidth='(-inf-0.34]' class=Iris-setosa 41 ==> petallength='(-inf-1.59]' 33    <conf:(0.8)> lift:(3.26) lev:(0.15) [22] conv:(3.43)\\
10. petalwidth='(-inf-0.34]' 41 ==> petallength='(-inf-1.59]' class=Iris-setosa 33    <conf:(0.8)> lift:(3.26) lev:(0.15) [22] conv:(3.43)\\



\section{Questão 05\\
  Considerando a base de dados do WEKA: Supermaket.arf, pede-se:
 }
\subsection{Anote os resultados usando o algoritmo APRIORI}
=== Run information ===

Scheme:       weka.associations.Apriori -N 10 -T 0 -C 0.9 -D 0.05 -U 1.0 -M 0.1 -S -1.0 -c -1
Relation:     supermarket
Instances:    4627
Attributes:   217
              [list of attributes omitted]
=== Associator model (full training set) ===


Apriori
=======

Minimum support: 0.15 (694 instances)
Minimum metric <confidence>: 0.9
Number of cycles performed: 17

Generated sets of large itemsets:

Size of set of large itemsets L(1): 44

Size of set of large itemsets L(2): 380

Size of set of large itemsets L(3): 910

Size of set of large itemsets L(4): 633

Size of set of large itemsets L(5): 105

Size of set of large itemsets L(6): 1

\subsection{Analise as regras. Veja o que você acha sobre elas e o que você pode fazer para melhorá-las}
Best rules found:

 1. biscuits=t frozen foods=t fruit=t total=high 788 ==> bread and cake=t 723    <conf:(0.92)> lift:(1.27) lev:(0.03) [155] conv:(3.35)\\
 2. baking needs=t biscuits=t fruit=t total=high 760 ==> bread and cake=t 696    <conf:(0.92)> lift:(1.27) lev:(0.03) [149] conv:(3.28)\\
 3. baking needs=t frozen foods=t fruit=t total=high 770 ==> bread and cake=t 705    <conf:(0.92)> lift:(1.27) lev:(0.03) [150] conv:(3.27)\\
 4. biscuits=t fruit=t vegetables=t total=high 815 ==> bread and cake=t 746    <conf:(0.92)> lift:(1.27) lev:(0.03) [159] conv:(3.26)\\
 5. party snack foods=t fruit=t total=high 854 ==> bread and cake=t 779    <conf:(0.91)> lift:(1.27) lev:(0.04) [164] conv:(3.15)\\
 6. biscuits=t frozen foods=t vegetables=t total=high 797 ==> bread and cake=t 725    <conf:(0.91)> lift:(1.26) lev:(0.03) [151] conv:(3.06)\\
 7. baking needs=t biscuits=t vegetables=t total=high 772 ==> bread and cake=t 701    <conf:(0.91)> lift:(1.26) lev:(0.03) [145] conv:(3.01)\\
 8. biscuits=t fruit=t total=high 954 ==> bread and cake=t 866    <conf:(0.91)> lift:(1.26) lev:(0.04) [179] conv:(3)\\
 9. frozen foods=t fruit=t vegetables=t total=high 834 ==> bread and cake=t 757    <conf:(0.91)> lift:(1.26) lev:(0.03) [156] conv:(3)\\
10. frozen foods=t fruit=t total=high 969 ==> bread and cake=t 877    <conf:(0.91)> lift:(1.26) lev:(0.04) [179] conv:(2.92)\\



\end{document}
