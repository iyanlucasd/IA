\documentclass[oneside]{article}            % Imprimir apenas frente
%\documentclass[doubleside]{diretrizes}        % Imprimir frente e verso

% Importações de pacotes
\usepackage[alf, abnt-emphasize=bf, recuo=0cm, abnt-etal-cite=2, abnt-etal-list=0]{abntex2cite}  % Citações padrão ABNT
\usepackage[utf8]{inputenc}                         % Acentuação direta
\usepackage[T1]{fontenc}                            % Codificação da fonte em 8 bits
\usepackage{graphicx}                               % Inserir figuras
\usepackage{amsfonts, amssymb, amsmath}             % Fonte e símbolos matemáticos
\usepackage{booktabs}                               % Comandos para tabelas
\usepackage{verbatim}                               % Texto é interpretado como escrito no documento
\usepackage{multirow, array}                        % Múltiplas linhas e colunas em tabelas
\usepackage{indentfirst}                            % Endenta o primeiro parágrafo de cada seção.
\usepackage{microtype}                              % Para melhorias de justificação?
\usepackage[algoruled, portuguese]{algorithm2e}     % Escrever algoritmos
\usepackage{float}                                  % Utilizado para criação de floats
\usepackage{times}                                  % Usa a fonte Times
\usepackage{lipsum} 

\title{Aula 1}
\author{Iyan Lucas}

\begin{document}
    \maketitle

\section{Cobertura de regra}

A cobertura por classe é basicamente a classe e quantas regras ela cobre (como se, de 100 amostras, ela só cobrisse 90), dividida pelo total da classe. 

A cobertura global é basicamente a cobertura de classe mas o divisor é o total de amostras da base.

\section{Matriz de Confusão e Tabela}

\subsection{Recall}
\begin{equation}
    \displaystyle\frac{Recall}{Sensibilidade} = \frac{VP}{VP+FN}
\end{equation}
\subsection{Precisão}

\begin{equation}
    Precisão = \frac{VP}{VP+FP}
\end{equation}



\end{document}