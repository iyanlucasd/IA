\documentclass[oneside]{article}            % Imprimir apenas frente
%\documentclass[doubleside]{diretrizes}        % Imprimir frente e verso

% Importações de pacotes
\usepackage[alf, abnt-emphasize=bf, recuo=0cm, abnt-etal-cite=2, abnt-etal-list=0]{abntex2cite}  % Citações padrão ABNT
\usepackage[utf8]{inputenc}                         % Acentuação direta
\usepackage[T1]{fontenc}                            % Codificação da fonte em 8 bits
\usepackage{graphicx}                               % Inserir figuras
\usepackage{amsfonts, amssymb, amsmath}             % Fonte e símbolos matemáticos
\usepackage{booktabs}                               % Comandos para tabelas
\usepackage{verbatim}                               % Texto é interpretado como escrito no documento
\usepackage{multirow, array}                        % Múltiplas linhas e colunas em tabelas
\usepackage{indentfirst}                            % Endenta o primeiro parágrafo de cada seção.
\usepackage{microtype}                              % Para melhorias de justificação?
\usepackage[algoruled, portuguese]{algorithm2e}     % Escrever algoritmos
\usepackage{float}                                  % Utilizado para criação de floats
\usepackage{times}                                  % Usa a fonte Times
\usepackage{lipsum} 

\title{Aula 1}
\author{Iyan Lucas}

\begin{document}
    \maketitle

\section{Cobertura de regra}

A cobertura por classe é basicamente a classe e quantas regras ela cobre (como se, de 100 amostras, ela só cobrisse 90), dividida pelo total da classe. 

A cobertura global é basicamente a cobertura de classe mas o divisor é o total de amostras da base.

\section{Matriz de Confusão e Tabela}

    \begin{itemize}
        \item Coluna: São os que são classificados como a classe
        \item Linha: os que realmente pertencem a classe
    \end{itemize}
    Exemplo:\\
    Na coluna A, foram classificados 8, 5 como A e 3 como B
    \begin{table}[h]
        \begin{tabular}{|lll|ll}
        \cline{1-3}
        \multicolumn{3}{|l|}{classificados}                  &  &  \\ \cline{1-3}
        \multicolumn{1}{|l|}{}  & \multicolumn{1}{l|}{A} & B &  &  \\ \cline{1-3}
        \multicolumn{1}{|l|}{A} & \multicolumn{1}{l|}{5} & 4 &  &  \\ \cline{1-3}
        \multicolumn{1}{|l|}{B} & \multicolumn{1}{l|}{3} & 2 &  &  \\ \cline{1-3}
        \end{tabular}
        \end{table}


\subsection{Recall}
\begin{equation}
    \displaystyle\frac{Recall}{Sensibilidade} = \frac{VP}{VP+FN}
\end{equation}
\subsection{Precisão}

\begin{equation}
    Precisão = \frac{VP}{VP+FP}
\end{equation}


\newpage
\section{Exercícios}
\begin{table}[h!]
    \begin{tabular}{|l|l|l|l|l|l|l|}
    \hline
      & VP    & FN    & FP    & VN    & Recall & Precision \\ \hline
    A & 10/15 & 5/15  & 10/61 & 51/61 & 10/15  & 10/20     \\ \hline
    B & 15/16 & 1/16  & 7/60  & 53/60 & 15/16  & 15/22     \\ \hline
    C & 30/45 & 15/45 & 4/31  & 27/31 & 30/45  & 30/34     \\ \hline
    \end{tabular}
    \end{table}

\end{document}