\documentclass[oneside]{article}            % Imprimir apenas frente
%\documentclass[doubleside]{diretrizes}        % Imprimir frente e verso

% Importações de pacotes
\usepackage[alf, abnt-emphasize=bf, recuo=0cm, abnt-etal-cite=2, abnt-etal-list=0]{abntex2cite}  % Citações padrão ABNT
\usepackage[utf8]{inputenc}                         % Acentuação direta
\usepackage[T1]{fontenc}                            % Codificação da fonte em 8 bits
\usepackage{graphicx}                               % Inserir figuras
\usepackage{amsfonts, amssymb, amsmath}             % Fonte e símbolos matemáticos
\usepackage{booktabs}                               % Comandos para tabelas
\usepackage{verbatim}                               % Texto é interpretado como escrito no documento
\usepackage{multirow, array}                        % Múltiplas linhas e colunas em tabelas
\usepackage{indentfirst}                            % Endenta o primeiro parágrafo de cada seção.
\usepackage{microtype}                              % Para melhorias de justificação?
\usepackage[algoruled, portuguese]{algorithm2e}     % Escrever algoritmos
\usepackage{float}                                  % Utilizado para criação de floats
\usepackage{times}                                  % Usa a fonte Times
\usepackage{lipsum} 

\title{Aula 2}
\author{Iyan Lucas}

\begin{document}
    \maketitle

    \section{Ganho e Entropia}
    \begin{eqnarray}
        ganho(atributo) = \Delta(classe) - \Delta(atributo)
    \end{eqnarray}\footnote{$\Delta$ é o símbolo da entropia.}
    A entropia da classe é o resultado da soma da entropia dos seus atributos.
    Para a entropia dos atributos, serão a negativa fração do atributo vezes o logarítmo deste mesmo atributos, exemplo:
    Supondo que existem 14 atributos, sendo 9 deles 'True' e 5 deles false
    \begin{equation}
        entropia = -(\frac{9}{14}) * \log_2(\frac{9}{14}) - (\frac{5}{14}) * \log_2(\frac{5}{14})
    \end{equation}
    Novamente na entropia de classe, precisaremos olhar a correspondência do atributo com a classe.
    Exemplo, se há uma correspondência do atributo e da classe:
    \begin{table}[h]
        \begin{tabular}{|l|l|}
        \hline
        Atributo & Classe \\ \hline
        Y        & Y      \\ \hline
        Y        & N      \\ \hline
        N        & N      \\ \hline
        N        & Y      \\ \hline
        Y        & Y      \\ \hline
        N        & Y      \\ \hline
        \end{tabular}
        \end{table}
        \begin{equation}
            entropia = Y(\frac{3}{6}*\Delta(\frac{2}{3}, \frac{1}{2})) + N(\frac{3}{6}*\Delta(\frac{1}{3}, \frac{2}{3}))
        \end{equation}
    \section{ID3 x J48(C45)}
    \begin{itemize}
        \item A poda
        \item Equação Diferente
        \item o ID3 só trabalha com atributos nominais e o C45 só com numéricos
        \item o ID3 não aceita dados auzentes
    \end{itemize}
    \section{Exercícios}
    \begin{table}[h]
        \begin{tabular}{l|l|l|l|}
        \cline{2-4}
                                & A  & B  & C  \\ \hline
        \multicolumn{1}{|l|}{A} & 10 & 0  & 0  \\ \hline
        \multicolumn{1}{|l|}{B} & 15 & 10 & 5  \\ \hline
        \multicolumn{1}{|l|}{C} & 5  & 0  & 30 \\ \hline
        \end{tabular}
        \end{table}

        \begin{table}[h]
            \begin{tabular}{l|l|l|l|l|l|l|}
            \cline{2-7}
                                    & TVP & TFN & TFP & TVN & Precision & Recall \\ \hline
            \multicolumn{1}{|l|}{A} &     &     &     &     &           &        \\ \hline
            \multicolumn{1}{|l|}{B} &     &     &     &     &           &        \\ \hline
            \multicolumn{1}{|l|}{C} &     &     &     &     &           &        \\ \hline
            \end{tabular}
            \end{table}
\end{document}