\documentclass[oneside]{article}            % Imprimir apenas frente
%\documentclass[doubleside]{diretrizes}        % Imprimir frente e verso

% Importações de pacotes
\usepackage[alf, abnt-emphasize=bf, recuo=0cm, abnt-etal-cite=2, abnt-etal-list=0]{abntex2cite}  % Citações padrão ABNT
\usepackage[utf8]{inputenc}                         % Acentuação direta
\usepackage[T1]{fontenc}                            % Codificação da fonte em 8 bits
\usepackage{graphicx}                               % Inserir figuras
\usepackage{amsfonts, amssymb, amsmath}             % Fonte e símbolos matemáticos
\usepackage{booktabs}                               % Comandos para tabelas
\usepackage{verbatim}                               % Texto é interpretado como escrito no documento
\usepackage{multirow, array}                        % Múltiplas linhas e colunas em tabelas
\usepackage{indentfirst}                            % Endenta o primeiro parágrafo de cada seção.
\usepackage{microtype}                              % Para melhorias de justificação?
\usepackage[algoruled, portuguese]{algorithm2e}     % Escrever algoritmos
\usepackage{float}                                  % Utilizado para criação de floats
\usepackage{times}                                  % Usa a fonte Times
\usepackage{lipsum} 

\title{Aula 4}
\author{Iyan Lucas}

\begin{document}
\maketitle
\section{Técnicas de Amostragem}
\subsection{Cross Validation}
O método de validação (o melhor método de validação) vai dividir o seu conjunto em x partes (folds).
E ele usa 9 dobras para treinar o modelo, e uma para validar o modelo. 
E as métricas de precisão e recall serão as médias dos valores das 10 dobras.
\subsection{Random Forest}
Da classe ensamble que ela foca em combinar classificadores.
Ela gera varias arvores e combina por voto majoritario, e valida pelo método bootstrap.
O bootstrap ele pega x exemplos de uma base (que podem ser repetidos) e valida com os exemplos que ele não validou.



\end{document}