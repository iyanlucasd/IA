\documentclass[oneside]{article}            % Imprimir apenas frente
%\documentclass[doubleside]{diretrizes}        % Imprimir frente e verso

% Importações de pacotes
\usepackage[alf, abnt-emphasize=bf, recuo=0cm, abnt-etal-cite=2, abnt-etal-list=0]{abntex2cite}  % Citações padrão ABNT
\usepackage[utf8]{inputenc}                         % Acentuação direta
\usepackage[T1]{fontenc}                            % Codificação da fonte em 8 bits
\usepackage{graphicx}                               % Inserir figuras
\usepackage{amsfonts, amssymb, amsmath}             % Fonte e símbolos matemáticos
\usepackage{booktabs}                               % Comandos para tabelas
\usepackage{verbatim}                               % Texto é interpretado como escrito no documento
\usepackage{multirow, array}                        % Múltiplas linhas e colunas em tabelas
\usepackage{indentfirst}                            % Endenta o primeiro parágrafo de cada seção.
\usepackage{microtype}                              % Para melhorias de justificação?
\usepackage[algoruled, portuguese]{algorithm2e}     % Escrever algoritmos
\usepackage{float}                                  % Utilizado para criação de floats
\usepackage{times}                                  % Usa a fonte Times
\usepackage{lipsum} 

\title{Aula 4}
\author{Iyan Lucas}

\begin{document}
\maketitle
\section{Questão 6}

\textit{Faça um resumo do artigo “Regras de Associação e suas Medidas de Interesse Objetivas e Subjetivas” que está em CANVAS/Módulos/Artigos}

O artigo busca revisar e, sucintamente descrever os métodos de pré-processamento.
Ele introduz explicando a importância do pré-processamento, afinal, a qualidade final da base de dados é imprescindível para o resultado final.
Desta forma, ele abrange métodos de como lidar com dados faltantes, nulos, faltantes e data cleaning.
Ele conclui explicitando como os dados do mundo real são inconstantes e inconsistente. 
Elas levam à uma eficiência, efetividade e papel importante na mineração de dados ou no data warehouse.

\end{document}